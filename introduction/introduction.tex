\chapter{Introduction}

% The introduction should summarise the subject area, the specific problem you are addressing, including key ideas for their solution, together with a summary of the project's main contributions. When detailing the contributions it is helpful to provide forward references to the section(s) of the report that provide the relevant technical details. The introduction should be aimed at an informed, but otherwise non-expert, reader. A good tip is to assume that all your assessors will read the abstract and introduction, whereas the more detailed technical sections may only be read by your first and second markers - it's therefore really important to get it right.

%% What is the problem
% Background on DR itself.

Diabetic retinopathy (DR) is an eye disease that commonly arises as a complication of diabetes. It is estimated that by the year 2045 693 million people worldwide will be living with diabetes. Of these, nearly all of those with type 1 and two-thirds of those with type 2 will be suffering some degree of retinopathy within 20 years of receiving their diagnosis \cite{Mathure014444}. Despite DR being preventable with early detection and intervention, this is made difficult by the fact that the early stages of DR show no symptoms and may only be identified by screening. However, the scalability of mass screening is severely limited by the availability of medical professionals, causing DR to remain as leading cause of blindness in the UK \cite{Liewe004015}. Currently, diagnosis requires manual inspection of retinal fundus images for the presence of abnormalities. This is a time consuming and error-prone process for ophthalmologists who, even when available, have been shown to be inconsistent \cite{DBLP:journals/corr/abs-1710-01711}. For this reason, automated and semi-automated techniques for DR diagnosis have been a popular topic of research, going back as far as .... As more sophisticated and powerful methods are developed, we come ever closer accessible screening for \emph{all} susceptible individuals.

The progression of DR is characterised by the formation of lesions on a patient's retina, the type and quantity of which indicate the severity of the disease. The ability to identify the precise locations of these lesions from the healthy parts of a retina ensures that the most appropriate and interpretable diagnosis possible can be given, as well as having uses in aiding surgical procedures. The task of labelling each pixel of an image to a class label is termed ``semantic segmentation''.

Developments in deep learning over the past decade have sparked a frenzy of breakthroughs in the field of medical imaging. Deep neural networks provide state-of-the-art models across a variety of domains, and continue to show even greater potential. This innovation is fuelled by greater quantities of data, allowing models to generalise to unseen examples. Indeed, in 2017 The Economist coined the phrase ``data is the new oil'', which has become a popular refrain to describe the value of having an abundance of data. This insatiable hunger for data is one of the primary obstacles in the lesion segmentation task, as large-scale, pixel-wise annotated datasets are scarce.

% The need for novel techniques.

An easy way to improve data diversity is by applying data augmentation techniques such as reflections, crops, rotations, and colour perturbations. However, these methods produce images that are highly correlated, and ultimately have low diversity. A more advanced method of image simulation has been to hand-craft complex mathematical models representing the human anatomy, from which images can be sampled. More recently, with the rise of data-driven techniques we have seen a paradigm shift away from this top-down approach to a bottom-up approach of learning the data distribution \emph{directly} from the data itself. This has been made possible by the introduction of Generative Adversarial Networks (GANs). The aim of this project is to leverage these generative models to produce realistic synthetic training data in large volumes.

Achieving large-scale data generation with arbitrary labels would yield significant improvements in the ability of neural networks to both semantically segment retinal fundus images, as well as assign image-level grades. Ultimately \emph{surpass} human ability. This project represents a step towards the goal of fully-automated retinal screening for the detection of diabetic retinopathy by presenting methods to combat the scarcity of data.